\chapter{Analysis and Design}
In this chapter, we will describe and justify the design of our
library and the thoughts and considerations that went into producing
the final design and ideas that was discarded along the way.

\section{Overall goal and success parameters}



% \section{Programming}
% Even though the eventual goal of the SME-library is to enable
% generated networks to execute
% The API exposed to the user of the SME library should be designed with
% a focus on balancing expressiveness and ability to be easily
% understandable....

% The API implemented by \cite{vinter2014synchronous} is heavily
% dependent of the highly dynamic nature of the Python programming
% language used to implement their prototype. Since our implementation
% is written in C++ we cannot directly mimic the python API in our
% implementation. Given the similarities between SME and CSP it seems
% obvious to let us inspire by

\section{Paralellization model}
A common way of parallelizing CSP-like networks is to use user-level
threads to represent a process. In comparison with OS-level threads,
user-level threads has a significantly lower overhead both with
regards to context switching penalty \cite{sung2002comparative} and
memory cost. Due to these limitations, implementing these kinds of
message passing systems using only OS-level threads are generally not
feasible. Therefore, user-level threads are used by other message
passing systems such as the C++CSP library\cite{brown2003introduction}
and the goroutines in the Go
language\cite{deshpandeanalysis}. However, implementing a user-level
threading library would add complexity to our program since we would
need to implement a scheduler, for scheduling processes on top of
OS-level threads.

Comparing, once again, to CSP, the concurrency in CSP is inherently
asynchronous while SME is entirely synchronous in nature. This means
that a CSP library needs to implement a scheduler which decides when
to give control to a process based on certain events, e.g. a process
wishing to communicate or a process receiving a message from another
process. \fxnote{Repeats some of previous paragraph}.

Our original design considerations for C++SME centered around such a
design, however, due to the enforced synchrony of SME we don't have
the same need to schedule processes ``intelligently'' since we know
that all processes needs to run during a cycle and all busses have to
propagate their values. This, in addition to he shared nothing
property of SME, allows us to specify a much simpler parallelization
model for SME compared to the techniques used by the aforementioned
message passing network implementations:

The basic idea that we base our design on is conceptually similar to a
classic producer-consumer setup. In our case, the work ``produced'' is
the processes to be executed, and the consumers are the threads
executing the processes. In this setup, a process is run simply by
calling a function, whereas user-level threads are usually implemented
using the \texttt{setcontext} and \texttt{getcontext} library
functions, which, while extremely fast, still causes a slightly larger
overhead compared to a simple function call. \fxnote{Switching between
  OS-level processes is actually really fast, since context-switching
  from one process to another usually only involves moving CPU
  registers, i.e. the stack pointer, to an appropriate location, TODO
  write something like that}

by simply letting a number of OS-threads run SME processes in a
worker-consumer like manner. This approach also make it simple enforce
the synchronicity properties of SME, since we know hen a process has
finished executing. Unlike

In this project, we have explored two different variations of this
basic idea. Both models are based on the idea described in the
previous paragraph. Our overall goal in parallelizing execution of
SME-networks is to minimize the amount of core idle time. We expect
the hereafter presented model to achieve this goal under different
circumstances \fxnote{different networks}.



\subsection{Pros and cons}
Pros: Shared nothing,forced synchrony 

Cons: Forced syncrhony 

\subsection{``Worker queue'' model}
This approach is similar to a classic producer-consumer model where we
have a number of workers which takes tasks off a circular queue and
executes the processes. The main advantage of this model is that it
allows processes with different execution times to ``interleave''
leading to a higher overall execution time. The primary problem of
this model is that we need to make the queue thread-safe. The locking
mechanism needed to do this isn't free, and could therefore come at a
significant cost when we're executing a network consisting of small
processes.

When shortness is needed, we will refer to this model as Model 1 or M1.

\subsection{Static orchestration model}
In this model, we assign separate queues to each thread of execution
and distribute (``orchestrate'') the processes amongst them. Due to
the properties of SME, this distribution of processes only needs to
happen once, before we start network execution. The main advantage of
this model is that eliminates any shared state in our network, and
therefore we don't need to consider the thread-safety of our
queues. This reduces the fixed cost of executing a process
significantly. This model, however, is more sensitive to uneven
distributions in process workloads. For instance, if we end up
assigning predominantly small processes to one core and large
processes to another, the core executing the small processes would be
left idle until the other core has finished executing it's part of the
cycle.

When shortness is needed, we will refer to this model as Motel 2 or M2.

Overall, we expect the latter model to have a significant advantage in
executing networks with small processes while the queue-locking cost
of the former model will perform better when executing networks with
large or unevenly distributed workloads since the queue-locking costs
owill be amortized and allow its process-interleaving ability to shine
through.

%Using OS-level threads for representing the SME processes was quickly
%ruled out during the design process. OS-level threads are severely
%limited in that the number of threads a process can have is highly
%limited \fxnote{Find reference for number of threads per process} and
%switching between OS-level threads is very expensive compared to, for
%instance, user-level threads. \cite{sung2002comparative}\fxnote{Find
%  more recent citation for the performance of OS vs. user-level threads}.

%Using user-level threads, on linux implemented by using the (now
%deprecated) \texttt{setjump, longjump} functions or the more current
%\texttt{ setcontext, getcontext} functions, is another possible way of
%implementing our networks. The notion of a user-level thread is highly
%compatible with our concept of a process, namely
%\fxnote{finish}. User-level threads is the method that is used to
%implement many CSP libraries, including C++CSP2.

%One important difference between the CSP and SME execution models is
%that CSP is asynchronous and event-driven in nature while SME is, as
%previously mentioned, entirely synchronous. This allows us to take a
%significantly more simple approach to scheduling threads compared to
%C++CSP. A parallel CSP library needs to schedule a process for
%execution based on events generated by other processes
%\cite{brown2003introduction}. In SME we know that every process needs
%to run \fxnote{This needs to be better described} during every ``clock
%cycle''. This greatly simplifies our implementation of multi-threaded
%SME.

The optimal way of showing 

\subsection{Identifying optimal process scheduling}
In order to determine the efficiency of various methods of process
scheduling we need to identify the optimality condition for our
process scheduling. An illustration of our threading model can be seen in
\cref{fig:suboptdist}. The green boxes represents processes while the
red boxes indicates core idle time. Notice, how we by redistributing
the processes across threads could reduce the idle-time of our
cores.


\section{Synchronizing cycles}
The main problem of executing an SME network is to make sure that the
cycles are synchronized, or that all the processes ``meet up'' at the
end of an execution cycle. While the problem of executing the actual
processes, due to the shared-nothing property of SME, is embarrassingly
parallel, the need for synchronization makes it not quite
so. Therefore, the time spent on synchronization will significantly
impact the overall performance of our network.

\fxnote{Talk about barriers somewhere, which probably describes some
  of what we're talking about more accurately}

\begin{enumerate}
\item In order to know when to stop executing, we need to count the
  number of cycles performed
\item In order to know when a cycle is complete, we need to keep track
  of the number of processes that has been executed.
\end{enumerate}

\subsection{Cost of synchronization}
The cost of synchronization arises from two areas


There are two places in the network execution where this
``accounting'' could be preformed. We could either place a ``guard''
around each ... An alternative way of performing synchronization is to
insert special p

``Naive'' way of doing would be to let each execution thread count the
number of processes that has been executed. The problem with this
approach is that it adds a fixed computational cost to each process
execution. This would become especially pronounced in model 1
which would require 

Furthermore, we would need to keep the progression of the network
execution as a globally shared state which would need to be protected
by locks when accessed by a thread. Both of these factors would
significantly limit the concurrent scalability of the parallel
execution.

One of the central parts of managing an execution cycle is how we
synchronize our threads before leaving each cycle phase
\fxnote{elaborate}. In order to maintain the previously described
synchrony property, ... Furthermore, the network execution must be
controlled so that we are able to stop the execution after a specified
number of cycles has completed.


An alternative method, which allows the 

\section{Implementing the queues}
An actual circular linked list where the last element points to the
first would be the most natural representation of the conceptual
circular queue that we just described. The usual advantage of using a
linked-list structure is that it allows for O(1) addition of
elements. The disadvantage if that element access is slower, even
though we would never need to actually traverse the list in order to
find a specific element, the cost associated with simply getting the
next element of linked list is not insignificant when performed enough
times \fxnote{crap}

A straight-forward array is much better suited for the task z7

\subsection{Locking primitices}
Classic locking mechanisms such as semaphores and mutexes needs no
introduction. We will, however, spend a little bit of time on
explaining the new kid on the block -- atomic operations.  Atomic
operations....

\subsection{Process orchestration}
\fxnote{This section doesn't really belong anywhere, remove or insert
  into other sections}
As we discussed in the previous section, the primary limiting factor
for our multi-threaded network is an uneven and suboptimal
distribution of processes across CPU-cores. If no attempt is made to
optimize process distribution, the order of process execution will
depend on the order of which processes are defined in the source
code. Due to \cref{noshare} and \cref{synchro} of SME
networks there is no scenario where it would be necessary or
beneficial for a programmer to exercise ultimate control over the
order of process execution. Therefore, maximizing CPU-core utilization
would be an unreasonable burden to put on the programmer, especially
since their optimization efforts would be specific to a certain number
of CPU-cores.

The optimal method and timing of process orchestration depends on the
dynamicity \fxnote{is predictability a better word?} of the work
performed by the network we are executing. A network where each
process performs a fixed amount of work per iteration will only need
to be orchestrated once, while a network where the workload of the
processes are variable will need to be continuously evaluated at
runtime in order to maintain our optimality condition. These various
methods will be discussed for the remainder of this section.


\begin{figure}
\centering
\includegraphics{figures/parallel}
\caption[Proposed SME parallelization model]{Example of suboptimal
  distribution of processes across processing threads. Green blocks
  represents processes while red blocks represents thread idle
  time. Threads are named $p_{i,j}$ where $i$ is the number of the
  thread the process has been assigned to and $f_i$ is the combined
  idle time for each thread. Threads are named $t_i$.}

\label{fig:suboptdist}

\end{figure}




%\fxnote{Discuss/investigate problem of
%  ``optimization-looping''. Imagine a network where the execution time
%  of each process is completely random. This would most likely trigger
%  a reorchestration after every iteration which may end up taking more
%  CPU-time than the actual process execution.}

\subsection{Round-robin orchestration}
\begin{figure}
\centering
\includegraphics{figures/roundrobin}

\caption[Round-robin orchestration]{Illustration of round-robin
  process orchestration. Progressive iterations are shown as
 
 increasing color intensity of arrows}

\label{fig:roundrobin}

\end{figure}


Processes will be executed on the first available core as seen in
\cref{fig:roundrobin}




%\section{Testing correctness/Ensuring correctness/SME compliance testing}}

% This should be somewhere else
%In order to check the correctness of our implementation of the
%SME-model we use a simple test network which can only be executed
%correctly if our implementation of SME is obedient to all of the
%required SME-propertoes. This test network will henceforth be referred
%to as the \textit{validator netowrk} or simply the \textit{validator}.

%Our test network consists of a validator process and $n$ plusone
%proccesses. A single bus carries an output value from the validator to
%all of the plusone processes...

%A number of error conditions that might occur during network execution
%include
%itemize...

%If executed correctly, the network will maintain the following
%invariant:

%We still need to make sure that our test network is adequately
%capable of detecting possible error conditions that may occur during
%execution. To do this, we use fuzzing. Fuzzing is a well known
%technique from security research in order to find vulnerabilities in
%software and works by, in a random and unpredictable manner, providing
%unexpected and invalid input to programs.

%In essence, the idea is to test the hypothesis(/statement) that if the
%validator is able to correctly detect a working network, it should
%also detect an incorrectly  working network.

%We adapt the fizzing approach to our use case by introducing a number
%of processes to our network which, in an unpredictable manner,
%simulates a number of failures

%\begin{enumerate}
%\item A process fails to increment it's input value for a single
%  iteration
%\item A process falls behind on a value increment
%\item A process skips returning a value for a single cycle
%\item A process repeatedly receives 0 from a channel
%\item A process simply stops 
%\end{enumerate}

%The observant reader may wonder how we're going to test for failures
%in the value propagation phase. Our proposition is that bus-related failures
%are fully covered by the above cases. For instance, several of the
%above cases are the direct equivalent of a failing (or delayed) bus propagation

%%% Local Variables:
%%% mode: latex
%%% TeX-master: "master"
%%% TeX-command-extra-options: "-enable-write18"
%%% End:
