\chapter{Conculsions}
% We have revealed some of the properties of SME. More real-world usage is still needed in order to decide on the optimal way to implement the network. We have given some options and shown that it is possible to support multiple execution models in the same framework.

We have successfully showed that it is possible to introduce a
parallelization model for SME which delivers significant speedups in
environments where sufficient hardware parallelism is available.

\noindent
The statically orchestrated model, in particular, proved to be extremely
successful and is capable of achieving near-linear speedups for certain
workloads. These results were made possible through our concept of
letting the execution flow "`self manage"' by using special processes.

\noindent
Our work list model, on the other hand, generally delivered a
disappointing performance. We attribute this to the synchronization
mechanism used which proved more expensive than anticipated.

\noindent
Based on the benchmarks that we have performed, we conclude that there
is a large difference between the speedups achieved by different
workloads. This raises the question of whether our results are
representative for SME networks implementing real-world systems.

\noindent
Looking ahead, our results opens up increased use of the SME model for
simulating hardware designs since we, through parallelization has been
able to significantly decrease the time required to execute SME
networks.

\noindent
Overall, our work has laid a foundation for additional exploration
into making parallelized SME implementations. Our results and
conclusions sets a clear direction for future work aiming to improve
this implementation even more.

%%% Local Variables:
%%% mode: latex
%%% TeX-master: "master"
%%% TeX-command-extra-options: "-enable-write18"
%%% End:
